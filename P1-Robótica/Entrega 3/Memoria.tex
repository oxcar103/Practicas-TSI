%%
% Plantilla de Memoria
% Modificación de una plantilla de Latex de Nicolas Diaz para adaptarla 
% al castellano y a las necesidades de escribir informática y matemáticas.
%
% Editada por: Mario Román
%
% License:
% CC BY-NC-SA 3.0 (http://creativecommons.org/licenses/by-nc-sa/3.0/)
%%

%%%%%%%%%%%%%%%%%%%%%
% Thin Sectioned Essay
% LaTeX Template
% Version 1.0 (3/8/13)
%
% This template has been downloaded from:
% http://www.LaTeXTemplates.com
%
% Original Author:
% Nicolas Diaz (nsdiaz@uc.cl) with extensive modifications by:
% Vel (vel@latextemplates.com)
%
% License:
% CC BY-NC-SA 3.0 (http://creativecommons.org/licenses/by-nc-sa/3.0/)
%
%%%%%%%%%%%%%%%%%%%%%

%----------------------------------------------------------------------------------------
%	PAQUETES Y CONFIGURACIÓN DEL DOCUMENTO
%----------------------------------------------------------------------------------------

%% Configuración del papel.
% microtype: Tipografía.
% mathpazo: Usa la fuente Palatino.
\documentclass[a4paper, 11pt]{article}
\usepackage[protrusion=true,expansion=true]{microtype}
\usepackage{mathpazo}

% Indentación de párrafos para Palatino
\setlength{\parindent}{0pt}
  \parskip=8pt
\linespread{1.05} % Change line spacing here, Palatino benefits from a slight increase by default

% Enlaces
\usepackage[hidelinks]{hyperref}

%% Castellano.
% noquoting: Permite uso de comillas no españolas.
% lcroman: Permite la enumeración con numerales romanos en minúscula.
% fontenc: Usa la fuente completa para que pueda copiarse correctamente del pdf.
\usepackage[spanish,es-noquoting,es-lcroman]{babel}
\usepackage[utf8]{inputenc}
\usepackage[T1]{fontenc}
\selectlanguage{spanish}


%% Gráficos
\usepackage{graphics,graphicx, float, url} % Required for including pictures
\usepackage{wrapfig} % Allows in-line images
\usepackage[usenames,dvipsnames]{color} % Coloring code
\usepackage{caption}
\usepackage{subcaption}

% Para algoritmos
\usepackage{algorithm}
\usepackage{algorithmic}
\usepackage{amsthm}

%% Matemáticas
\usepackage{amsmath}


%% Bibliografía
\makeatletter
\renewcommand\@biblabel[1]{\textbf{#1.}} % Change the square brackets for each bibliography item from '[1]' to '1.'
\renewcommand{\@listI}{\itemsep=0pt} % Reduce the space between items in the itemize and enumerate environments and the bibliography



%----------------------------------------------------------------------------------------
%	TÍTULO
%----------------------------------------------------------------------------------------
% Configuraciones para el título.
% El título no debe editarse aquí.
\renewcommand{\maketitle}{
  \begin{flushright} % Right align
  
  {\LARGE\@title} % Increase the font size of the title
  
  \vspace{50pt} % Some vertical space between the title and author name
  
  {\large\@author} % Author name
  \\\@date % Date
  \vspace{40pt} % Some vertical space between the author block and abstract
  \end{flushright}
}

% Título
\title{\textbf{Prácticas ROS: Robótica}\\ % Title
Entrega 3: Navegación global} % Subtitle

\author{\textsc{Óscar Bermúdez Garrido,\\Iván Calle Gil,\\ Luis Castro Martín,\\ Eva Mª González García} % Author
\\{\textit{Universidad de Granada}}} % Institution

\date{\today} % Date



%----------------------------------------------------------------------------------------
%	DOCUMENTO
%----------------------------------------------------------------------------------------

\begin{document}

\maketitle % Print the title section

% Resumen (Descomentar para usarlo)
\renewcommand{\abstractname}{Resumen} % Uncomment to change the name of the abstract to something else
\begin{abstract}
	En esta práctica, procederemos a implementar el comportamiento de un robot a través de un mapa
	tratando de ir desde su posición inicial a otra posición que se le indicará inicialmente como
	destino u objetivo.
	
	Para ello, usaremos el algoritmo A* para desplazarse por el mapa evitando, chocar contra los diversos
	obstáculos que se pueda ir encontrando (paredes, cajas, \dots) a lo largo de su camino antes de
	lograr llegar a su objetivo.

	Una vez conseguido este objetivo, nos propondremos mejorar la ejecución del mismo, es decir, buscaremos
	conseguir a la vez que se minimice el tiempo del recorrido y se maximice la seguridad del mismo.
	
	A modo de ejemplo, hemos incluido algunas pruebas sobre distintos mapas.
\end{abstract}

% Palabras clave
%\hspace*{3,6mm}\textit{Keywords:} lorem , ipsum , dolor , sit amet , lectus % Keywords
%\vspace{30pt} % Some vertical space between the abstract and first section

% Índice
{\parskip=2pt
  \tableofcontents
}
\pagebreak

%% Inicio del documento

\section{Implementación del algoritmo A*}
	Para la implementación del \textbf{algoritmo A*}, nos basamos en el material proporcionado en el
	temario de teoría, así como en el \href{http://theory.stanford.edu/~amitp/GameProgramming/}
	{\textit{material proporcionado}}.

	Para ello, necesitaremos transformar el índice de celdas a coordenadas del mundo. Esta implementación,
	la haremos tal y siguiendo las indicaciones que se nos proporcionan para la consecución de dicho fin
	en la \href{http://docs.ros.org/indigo/api/costmap_2d/html/classcostmap__2d_1_1Costmap2D.html}
	{\textit{documentación de ROS}}.
		

\section{Mejora del algoritmo A*}
	Para mejorar el \textbf{algoritmo A*}, optamos por introducir la variación de peso mediante la
	ponderación por la máxima distancia disponible en el mapa.
	
	Esto es, el peso de nuestra variación sería:
	$$\omega = \frac{\texttt{distancia al objetivo}}{\texttt{distancia máxima del mapa}} \cdot 3 + 1$$

\section{Pruebas en distintos mapas}
	\subsection{laberinto}
		\subsubsection{Algoritmo A*}
			En esta ejecución, se obtuvo un tiempo de 12.0125 s. con el siguiente recorrido:
			\begin{figure}[H]
				\centering
				\includegraphics[width=6cm]{Pruebas/A*/laberinto/laberinto(1)-1}
				\includegraphics[width=6cm]{Pruebas/A*/laberinto/laberinto(1)-2}
				\includegraphics[width=6cm]{Pruebas/A*/laberinto/laberinto(1)-3}
				\includegraphics[width=6cm]{Pruebas/A*/laberinto/laberinto(1)-4}
				\includegraphics[width=6cm]{Pruebas/A*/laberinto/laberinto(1)-5}
				\caption{Recorrido del robot.}
				\label{A-lab}
			\end{figure}
			
		\subsubsection{Mejora del algoritmo A*}
			En esta ejecución, se aprecia una pequeña mejora, dando como resultados un tiempo de 10.9578s.
			y el recorrido: 
			
			\begin{figure}[H]
				\centering
				\includegraphics[width=6cm]{Pruebas/Mejora-A*/laberinto/laberinto(2)-1}
				\includegraphics[width=6cm]{Pruebas/Mejora-A*/laberinto/laberinto(2)-2}
				\includegraphics[width=6cm]{Pruebas/Mejora-A*/laberinto/laberinto(2)-3}
				\includegraphics[width=6cm]{Pruebas/Mejora-A*/laberinto/laberinto(2)-4}
				\includegraphics[width=6cm]{Pruebas/Mejora-A*/laberinto/laberinto(2)-5}
				\caption{Recorrido del robot.}
				\label{MA-lab}
			\end{figure}
	\subsection{simplerooms}
		\subsubsection{Algoritmo A*}
			En esta ejecución, se obtuvo un tiempo de 32.4385 s. siguiendo el camino:
			\begin{figure}[H]
				\centering
				\includegraphics[width=6cm]{Pruebas/A*/simplerooms/simplerooms(1)-1}
				\includegraphics[width=6cm]{Pruebas/A*/simplerooms/simplerooms(1)-2}
				\includegraphics[width=6cm]{Pruebas/A*/simplerooms/simplerooms(1)-3}
				\includegraphics[width=6cm]{Pruebas/A*/simplerooms/simplerooms(1)-4}
				\includegraphics[width=6cm]{Pruebas/A*/simplerooms/simplerooms(1)-5}
				\includegraphics[width=6cm]{Pruebas/A*/simplerooms/simplerooms(1)-6}
				\includegraphics[width=6cm]{Pruebas/A*/simplerooms/simplerooms(1)-7}
				\caption{Recorrido del robot.}
				\label{A-sim}
			\end{figure}
			
		\subsubsection{Mejora del algoritmo A*}
			En esta ejecución, se aprecia un ligero empeoramiento, con un tiempo de 32.7008 s. y ruta: 

			\begin{figure}[H]
				\centering
				\includegraphics[width=6cm]{Pruebas/Mejora-A*/simplerooms/simplerooms(2)-1}
				\includegraphics[width=6cm]{Pruebas/Mejora-A*/simplerooms/simplerooms(2)-2}
				\includegraphics[width=6cm]{Pruebas/Mejora-A*/simplerooms/simplerooms(2)-3}
				\includegraphics[width=6cm]{Pruebas/Mejora-A*/simplerooms/simplerooms(2)-4}
				\includegraphics[width=6cm]{Pruebas/Mejora-A*/simplerooms/simplerooms(2)-5}
				\includegraphics[width=6cm]{Pruebas/Mejora-A*/simplerooms/simplerooms(2)-6}
				\includegraphics[width=6cm]{Pruebas/Mejora-A*/simplerooms/simplerooms(2)-7}
				\includegraphics[width=6cm]{Pruebas/Mejora-A*/simplerooms/simplerooms(2)-8}
				\caption{Recorrido del robot.}
				\label{MA-sim}
			\end{figure}
			
	\subsection{willow}
		\subsubsection{Algoritmo A*}
			En esta ejecución, el tiempo fue 25.7184 s. con el recorrido:
			\begin{figure}[H]
				\centering
				\includegraphics[width=6cm]{Pruebas/A*/willow/willow(1)-1}
				\includegraphics[width=6cm]{Pruebas/A*/willow/willow(1)-2}
				\includegraphics[width=6cm]{Pruebas/A*/willow/willow(1)-3}
				\caption{Recorrido del robot.}
				\label{A-wil}
			\end{figure}
			
		\subsubsection{Mejora del algoritmo A*}
			En esta ejecución, se aprecia una gran mejora con la incorporación de los pesos, dando como
			resultados un tiempo de 13.5573 s. y el recorrido: 
			
			\begin{figure}[H]
				\centering
				\includegraphics[width=6cm]{Pruebas/Mejora-A*/willow/willow(2)-1}
				\includegraphics[width=6cm]{Pruebas/Mejora-A*/willow/willow(2)-2}
				\includegraphics[width=6cm]{Pruebas/Mejora-A*/willow/willow(2)-3}
				\includegraphics[width=6cm]{Pruebas/Mejora-A*/willow/willow(2)-4}
				\includegraphics[width=6cm]{Pruebas/Mejora-A*/willow/willow(2)-5}
				\caption{Recorrido del robot.}
				\label{MA-wil}
			\end{figure}
	
\end{document}
