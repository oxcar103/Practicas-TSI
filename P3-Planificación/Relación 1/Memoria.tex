%%
% Plantilla de Memoria
% Modificación de una plantilla de Latex de Nicolas Diaz para adaptarla 
% al castellano y a las necesidades de escribir informática y matemáticas.
%
% Editada por: Mario Román
%
% License:
% CC BY-NC-SA 3.0 (http://creativecommons.org/licenses/by-nc-sa/3.0/)
%%

%%%%%%%%%%%%%%%%%%%%%
% Thin Sectioned Essay
% LaTeX Template
% Version 1.0 (3/8/13)
%
% This template has been downloaded from:
% http://www.LaTeXTemplates.com
%
% Original Author:
% Nicolas Diaz (nsdiaz@uc.cl) with extensive modifications by:
% Vel (vel@latextemplates.com)
%
% License:
% CC BY-NC-SA 3.0 (http://creativecommons.org/licenses/by-nc-sa/3.0/)
%
%%%%%%%%%%%%%%%%%%%%%

%----------------------------------------------------------------------------------------
%	PAQUETES Y CONFIGURACIÓN DEL DOCUMENTO
%----------------------------------------------------------------------------------------

%% Configuración del papel.
% microtype: Tipografía.
% mathpazo: Usa la fuente Palatino.
\documentclass[a4paper, 11pt]{article}
\usepackage[protrusion=true,expansion=true]{microtype}
\usepackage{mathpazo}

% Indentación de párrafos para Palatino
\setlength{\parindent}{0pt}
  \parskip=8pt
\linespread{1.05} % Change line spacing here, Palatino benefits from a slight increase by default

% Enlaces
\usepackage[hidelinks]{hyperref}

%% Castellano.
% noquoting: Permite uso de comillas no españolas.
% lcroman: Permite la enumeración con numerales romanos en minúscula.
% fontenc: Usa la fuente completa para que pueda copiarse correctamente del pdf.
\usepackage[spanish,es-noquoting,es-lcroman]{babel}
\usepackage[utf8]{inputenc}
\usepackage[T1]{fontenc}
\selectlanguage{spanish}

%% Gráficos
\usepackage{graphics,graphicx, float, url} % Required for including pictures
\usepackage{wrapfig} % Allows in-line images
\usepackage[usenames,dvipsnames]{color} % Coloring code
\usepackage{caption}
\usepackage{subcaption}

% Para algoritmos
\usepackage{algorithm}
\usepackage{algorithmic}
\usepackage{amsthm}

\makeatletter

%----------------------------------------------------------------------------------------
%	TÍTULO
%----------------------------------------------------------------------------------------
% Configuraciones para el título.
% El título no debe editarse aquí.
\renewcommand{\maketitle}{
  \begin{flushright} % Right align
  
  {\LARGE\@title} % Increase the font size of the title
  
  \vspace{50pt} % Some vertical space between the title and author name
  
  {\large\@author} % Author name
  \\\@date % Date
  \vspace{40pt} % Some vertical space between the author block and abstract
  \end{flushright}
}

% Título
\title{\textbf{Prácticas PDDL: Planificación}\\ % Title
Entrega 1: Dominios y problemas de planificación clásica en PDDL} % Subtitle

\author{\textsc{Óscar Bermúdez Garrido} % Author
\\{\textit{Universidad de Granada}}} % Institution

\date{\today} % Date

%----------------------------------------------------------------------------------------
%	DOCUMENTO
%----------------------------------------------------------------------------------------

\begin{document}

\maketitle % Print the title section

% Resumen (Descomentar para usarlo)
\renewcommand{\abstractname}{Resumen} % Uncomment to change the name of the abstract to something else
\begin{abstract}
	En esta práctica, nos presentan una aventura gráfica y tenemos que modelar sus reglas, objetivos, así
	como el comportamiento del jugador de forma deliberativa para lograr alcanzar las metas que se le vayan
	proponiendo de la forma más óptima posible.
\end{abstract}

% Índice
{\parskip=2pt
  \tableofcontents
}
\pagebreak

%% Inicio del documento

\section{Mapa del mundo}
	Por la complejidad de las 25 zonas, me fue necesario el dibujo de un mapa para testear (que
	posteriormente reutilicé) para facilitarme la creación del archivo de problema y no dejar
	zonas aislada.
	
	Las posibles zonas aisladas se pueden producir por:
	\begin{itemize}
		\item Establecer dos o más zonas no conectadas.
		\item Ubicar precipios en zonas que necesitaban ser transitables.
		\item De forma más complicada, colocar bosques o agua en zonas que necesiten ser atravesadas
		sin colocar el objeto correspondiente para atravesarlos de forma accesible.
	\end{itemize} 
	
	El mapa utilizado fue el siguiente:
	\begin{figure}[H]
		\centering
		\includegraphics[width=15cm]{BelkanWorld.png}
		\caption{Mapa de los mundos de Belkan.}
		\label{world}
	\end{figure}

\section{Características básicas}
	En este apartado, se configuran las características más básicas del juego, dejando para apartados
	posteriores la implementación de las reglas, limitaciones y optimizaciones.
	
	\subsection{Objetos del mundo}
	Para la implementación de los objetos del mundo, usaremos la sintaxis de PDDL dada por:
	\begin{verbatim}
	(:types jugador
	        personaje
	        objeto
	        zona
	)
	\end{verbatim}
	
	Aunque se podría haber optado por definir cada uno de los personajes y objetos en este
	apartado, considero más adecuado la inicialización de los distintos personajes y objetos
	dentro del archivo de problema.
	
	
\end{document}
