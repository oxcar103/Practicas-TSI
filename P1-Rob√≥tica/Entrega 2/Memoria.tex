%%
% Plantilla de Memoria
% Modificación de una plantilla de Latex de Nicolas Diaz para adaptarla 
% al castellano y a las necesidades de escribir informática y matemáticas.
%
% Editada por: Mario Román
%
% License:
% CC BY-NC-SA 3.0 (http://creativecommons.org/licenses/by-nc-sa/3.0/)
%%

%%%%%%%%%%%%%%%%%%%%%
% Thin Sectioned Essay
% LaTeX Template
% Version 1.0 (3/8/13)
%
% This template has been downloaded from:
% http://www.LaTeXTemplates.com
%
% Original Author:
% Nicolas Diaz (nsdiaz@uc.cl) with extensive modifications by:
% Vel (vel@latextemplates.com)
%
% License:
% CC BY-NC-SA 3.0 (http://creativecommons.org/licenses/by-nc-sa/3.0/)
%
%%%%%%%%%%%%%%%%%%%%%

%----------------------------------------------------------------------------------------
%	PAQUETES Y CONFIGURACIÓN DEL DOCUMENTO
%----------------------------------------------------------------------------------------

%% Configuración del papel.
% microtype: Tipografía.
% mathpazo: Usa la fuente Palatino.
\documentclass[a4paper, 11pt]{article}
\usepackage[protrusion=true,expansion=true]{microtype}
\usepackage{mathpazo}

% Indentación de párrafos para Palatino
\setlength{\parindent}{0pt}
  \parskip=8pt
\linespread{1.05} % Change line spacing here, Palatino benefits from a slight increase by default


%% Castellano.
% noquoting: Permite uso de comillas no españolas.
% lcroman: Permite la enumeración con numerales romanos en minúscula.
% fontenc: Usa la fuente completa para que pueda copiarse correctamente del pdf.
\usepackage[spanish,es-noquoting,es-lcroman]{babel}
\usepackage[utf8]{inputenc}
\usepackage[T1]{fontenc}
\selectlanguage{spanish}


%% Gráficos
\usepackage{graphicx} % Required for including pictures
\usepackage{wrapfig} % Allows in-line images
\usepackage[usenames,dvipsnames]{color} % Coloring code


%% Matemáticas
\usepackage{amsmath}


%% Bibliografía
\makeatletter
\renewcommand\@biblabel[1]{\textbf{#1.}} % Change the square brackets for each bibliography item from '[1]' to '1.'
\renewcommand{\@listI}{\itemsep=0pt} % Reduce the space between items in the itemize and enumerate environments and the bibliography



%----------------------------------------------------------------------------------------
%	TÍTULO
%----------------------------------------------------------------------------------------
% Configuraciones para el título.
% El título no debe editarse aquí.
\renewcommand{\maketitle}{
  \begin{flushright} % Right align
  
  {\LARGE\@title} % Increase the font size of the title
  
  \vspace{50pt} % Some vertical space between the title and author name
  
  {\large\@author} % Author name
  \\\@date % Date
  \vspace{40pt} % Some vertical space between the author block and abstract
  \end{flushright}
}

% Título
\title{\textbf{Prácticas ROS: Robótica}\\ % Title
Entrega 2: Navegación local} % Subtitle

\author{\textsc{Óscar Bermúdez Garrido,\\Iván Calle Gil,\\ Luis Castro Martín,\\ Eva Mª González García} % Author
\\{\textit{Universidad de Granada}}} % Institution

\date{\today} % Date



%----------------------------------------------------------------------------------------
%	DOCUMENTO
%----------------------------------------------------------------------------------------

\begin{document}

\maketitle % Print the title section

% Resumen (Descomentar para usarlo)
\renewcommand{\abstractname}{Resumen} % Uncomment to change the name of the abstract to something else
\begin{abstract}
	
\end{abstract}

% Palabras clave
%\hspace*{3,6mm}\textit{Keywords:} lorem , ipsum , dolor , sit amet , lectus % Keywords
%\vspace{30pt} % Some vertical space between the abstract and first section

% Índice
{\parskip=2pt
  \tableofcontents
}
\pagebreak

%% Inicio del documento

\section{Campos de potencial}
	En la función \textbf{getOneDeltaRepulsivo}, hemos incorporado las fórmulas respectivas al cálculo
	de campos de potencial vistas en clase para lograr que nuestro robot evita satisfactoriamente los
	obstáculos que se hayan dispersos por el mapa. Hemos optado por utilizar un valor de 100  para
	la fuerza atractiva como un valor lo suficientemente grande\footnote{Nótese que el máximo de dicha
	fuerza es $\alpha \cdot s$, con $\alpha$ la intensidad del campo atractivo y $s$ su extensión.}.
	
	En función a ella, podemos completar la función \textbf{setTotalRepulsivo} de forma que calcule la
	suma de las fuerzas repulsivas de los obstáculos detectados. Una vez implementado el campo de repulsivo,
	lo usamos en el objeto \textit{planner} del servidor.
	
	Para alterar el campo de visión del escaneo láser de nuestro robot, modificamos los valores de las
	constantes \textbf{MIN\_SCAN\_ANGLE\_RAD} y \textbf{MAX\_SCAN\_ANGLE\_RAD}. Inicialmente, aparecían
	con un valor de \textit{-30.0/180*M\_PI} y \textit{+30.0/180*M\_PI} respectivamente, esto es, nos
	otorgaban un ángulo de visión de 60º centrado en el frente.
	
	Como queremos que nuestro campo de visión se incremente hasta alcanzar los 270º, debemos modificarlos
	por los valores de \textit{-135.0/180*M\_PI} para la constante \textbf{MIN\_SCAN\_ANGLE\_RAD} y de
	\textit{+135.0/180*M\_PI} para la que se llama \textbf{MAX\_SCAN\_ANGLE\_RAD}.
	
	Aunque hemos optado por repartirlos de forma uniforme y que quede la visión centrada en el frente,
	se podrían convertir estas constantes en variables para poder cambiar el centro de la visión en
	determinados casos que pudiesen ser de interés, como por ejemplo, al tomar una curvatura para esquivar
	un obstáculo.

	Respecto a la corrección del exceso de tiempo que el robot se toma para detenerse al llegar a su
	objetivo, nos basta con colocar un tope en la reducción de la fuerza atractiva en función de la
	distancia a nuestro objetivo que consideraba inicialmente la fórmula utilizaba para la implementación
	del cálculos de los campos potenciales. Así, obtenemos que cuando está lo suficientemente cerca de
	lograr alcanzarlo, se mueve cada vez más despacio hasta que, finalmente, lo alcanza y se detiene.
	
	El robot presenta un comportamiento suicida al intentar evitar un obstáculo, por ejemplo, mediante
	un giro. Esto se produce porque al realizar el cálculo del módulo de la fuerza obtiene un valor
	excesivamente grande. Éste repercute en la velocidad incrementándola muchísimo. El resultado, que
	el robot acaba estrellándose contra un muro a gran velocidad.
	
	Para corregir este comportamiento errático y nada deseable del robot debemos detectar un cambio en
	la velocidad angular superior al normal y actuar en consecuencia frenando temporalmente, si fuese
	necesario, la velocidad lineal permitiendo a nuestro dispositivo no tripulado disponer del tiempo
	que necesita para realizar el giro en el que está inmerso de una manera satisfactoria.
	
\section{Escapar de mínimos locales}

\section{Pruebas en distintos mapas}
	

\end{document}
